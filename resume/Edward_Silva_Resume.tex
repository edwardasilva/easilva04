\documentclass[11pt]{article}       % set main text size
\usepackage{geometry}
\geometry{letterpaper,              % set paper size to letter
top=0.5in,                          % specify top page margin
bottom=0.5in,                       % specify bottom page margin
left=0.5in,                         % specify left page margin
right=0.5in}                        % specify right page margin

\usepackage{XCharter}               % set font
\usepackage[T1]{fontenc}            % output encoding
\usepackage[utf8]{inputenc}         % input encoding
\usepackage{enumitem}               % enable lists for bullet points: itemize and \item
\usepackage[hidelinks]{hyperref}    % format hyperlinks
\usepackage{titlesec}               % enable section title customization
\raggedright                        % disable text justification
\pagestyle{empty}                   % disable page numbering

% ensure PDF output will be all-Unicode and machine-readable
\input{glyphtounicode}
\pdfgentounicode=1

% format section headings: bolding, size, white space above and below
\titleformat{\section}{\bfseries\large}{}{0pt}{}[\vspace{1pt}\titlerule\vspace{-10pt}]

% format bullet points: size, white space above and below, white space between bullets
\renewcommand\labelitemi{$\vcenter{\hbox{\small$\bullet$}}$}
\setlist[itemize]{itemsep=-2pt, leftmargin=12pt}

% resume starts here
\begin{document}

% name
\centerline{\huge Edward Silva}
\vspace{5pt}

% contact information
\centerline{
\href{https://www.linkedin.com/in/edwardasilva/}{Linkedin.com/in/edwardasilva}
| \href{https://easilva.com}{easilva.com}
| \href{mailto:easilva@mines.edu}{easilva@mines.edu} 
| \href{tel:7027207735}{(702) 720-7735}
}

\vspace{-14pt}
% experience section
\section*{Experience}
\vspace{5pt}
\textbf{Co-op Intern, Electrical Design, }{Jordan and Skala Engineers} -- Denver, CO \hfill January -- June 2025 \\
\vspace{-6.5pt}
\begin{itemize}
  \item Contributed to electrical design of 20+ multi-unit residential and specialty building developments, spanning initial takeoffs, layout design, riser diagrams, NEC verification, and QC review.
  \item Developed proficiency in Autodesk Revit and MEP AutoCAD, strategically placing electrical receptacles, lighting, and circuits to ensure NEC compliance and practical, user-centered functionality.
  \item Performed circuit loading and voltage drop calculations, balancing panel schedules and selecting appropriate breakers to ensure safety, reliability, and adherence to regulatory standards.
  \item Utilized existing automation between Revit/CAD layouts and Excel tracking sheets to streamline design documentation processes and reduce manual errors.
  \item Collaborated closely with supervisors and cross-disciplinary teams (Mechanical, Plumbing), documenting client interactions and team meetings to improve project coordination and team efficiency.
\end{itemize}

\textbf{Undergraduate Researcher, }{\href{https://www.epowerhubs.com/home}{ePower Hubs Research Lab}} -- Golden, CO \hfill June -- December 2024 \\
\vspace{-6.5pt}
\begin{itemize}
  \item Independently conducted literature reviews on sensor systems and wind farm-level control strategies, focusing on offshore integration with variable voltage, power, and frequency constraints.
  \item Synthesized findings into multiple internal reports using LaTeX, contributing to cost-reduction strategies in wind farm grid maintenance, design, and power grid integration.
  \item Provided insights that influenced the direction of ongoing research led by a faculty advisor, shaping the lab's approach to offshore wind system modeling.
\end{itemize}

% projects section
\vspace{-18pt}
\section*{Projects}
\vspace{5pt}

\textbf{Dual-Axis Solar Tracker, }{Python, Arduino, \href{https://github.com/edwardasilva/SolarPanelProject}{Github}} \hfill{August -- October 2024} \\
\vspace{-6.5pt}
\begin{itemize}
  \item Designed and built a dual-axis solar tracking prototype using Arduino-controlled servos and photoresistor-based voltage divider circuits to maximize solar exposure.
  \item Wrote a custom tracking algorithm from scratch to identify the brightest point in the sky through light intensity sampling, enabling precise pitch and yaw adjustments.
  \item Utilized a Raspberry Pi as the system's central controller, handling logic flow and interfacing with the Arduino to execute real-time motor positioning.
\end{itemize}


% education section
\vspace{-18 pt}
\section*{Education}
\vspace{5pt}

\textbf{\href{https://www.mines.edu/}{Colorado School of Mines}}, \textbf{GPA:} 3.44  \hfill May 2026\\
\textbf{\href{https://electrical.mines.edu/undergraduate-program/}{BS, Electrical Engineering}} -- Controls \& Signal Processing  \\
\href{https://cs.mines.edu/csmines-minors-and-areas-of-special-interest/}{Minor, Computer Science} -- Algorithm Design\\
\textbf{Courses:} Advanced Control Systems, Signals \& Systems, Embedded Systems, Software Engineering
\textbf{Certifications:} Microsoft Technical Associate (MTA): Python \& Java Programming, MATLAB Machine Learning

% skills section
\section*{Skills}
\vspace{5pt}

\textbf{Programming Languages:} Java, Python, Verilog, C, C++, C\#, RISC-V Assembly, Bash, MATLAB, JavaScript \\
\textbf{Technology:} SSH, Linux OS (Ubuntu), Raspberry Pi, Arduino \\
\textbf{Software:} Autodesk Revit, MEP AutoCAD, VS Code, GitHub \\

\end{document}
